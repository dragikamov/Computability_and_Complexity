\documentclass[a4paper]{article}
\usepackage[pdftex]{hyperref}
\usepackage[latin1]{inputenc}
\usepackage[english]{babel}
\usepackage{a4wide}
\usepackage{amsmath}
\usepackage{amssymb}
\usepackage{algorithmic}
\usepackage{algorithm}
\usepackage{ifthen}
\usepackage{listings}
\usepackage{enumitem}
% move the asterisk at the right position
\lstset{basicstyle=\ttfamily,tabsize=4,literate={*}{${}^*{}$}1}
%\lstset{language=C,basicstyle=\ttfamily}
\usepackage{moreverb}
\usepackage{palatino}
\usepackage{multicol}
\usepackage{tabularx}
\usepackage{comment}
\usepackage{verbatim}
\usepackage{color}

% Because of an error on line 41 I added this
\usepackage{graphicx}

% Used for drawing DFAs and NFAs
\usepackage{tikz}
\usetikzlibrary{automata, positioning}

% Defined checkmark sign
\def\checkmark{\tikz\fill[scale=0.4](0,.35) -- (.25,0) -- (1,.7) -- (.25,.15) -- cycle;}

% Table coloring
\usepackage{color, colortbl}
\usepackage[first=0,last=9]{lcg}

\DeclareFontEncoding{LS1}{}{}
\DeclareFontSubstitution{LS1}{stix}{m}{n}
\DeclareSymbolFont{symbolsstix}{LS1}{stixscr}{m}{n}
\SetSymbolFont{symbolsstix}{bold}{LS1}{stixscr}{b}{n}
\DeclareMathSymbol{\mathvisiblespace}{0}{symbolsstix}{"B6}

% Tab
\newcommand\tab[1][1.15cm]{\hspace*{#1}}

%% pdflatex?
\newif\ifpdf
\ifx\pdfoutput\undefined
\pdffalse % we are not running PDFLaTeX
\else
\pdfoutput=1 % we are running PDFLaTeX
\pdftrue
\fi
%\ifpdf
%\usepackage[pdftex]{graphicx}
%\else
%\usepackage{graphicx}
%\fi
\ifpdf
\DeclareGraphicsExtensions{.pdf, .jpg}
\else
\DeclareGraphicsExtensions{.eps, .jpg}
\fi

\parindent=0cm
\parskip=0cm

\setlength{\columnseprule}{0.4pt}
\addtolength{\columnsep}{2pt}

\addtolength{\textheight}{5.5cm}
\addtolength{\topmargin}{-26mm}
\pagestyle{empty}

%%
%% Sheet setup
%% 
\newcommand{\coursename}{Computability and Complexity}
\newcommand{\courseno}{CO21-320352}
 
\newcommand{\sheettitle}{Homework}
\newcommand{\mytitle}{}
\newcommand{\mytoday}{March 26th, 2019}

% Current Assignment number
\newcounter{assignmentno}
\setcounter{assignmentno}{6}

% Current Problem number, should always start at 1
\newcounter{problemno}
\setcounter{problemno}{1}

%%
%% problem and bonus environment
%%
\newcounter{probcalc}
\newcommand{\exercise}[2]{
  \pagebreak[2]
  \setcounter{probcalc}{#2}
  ~\\
  {\large \textbf{Exercise \arabic{problemno}} \hspace{0.2cm}\textit{#1}} \refstepcounter{problemno}\vspace{2pt}\\}

\newcommand{\bonus}[2]{
  \pagebreak[2]
  \setcounter{probcalc}{#2}
  ~\\
  {\large \textbf{Bonus Problem \textcolor{blue}{\arabic{assignmentno}}.\textcolor{blue}{\arabic{problemno}}} \hspace{0.2cm}\textit{#1}} \refstepcounter{problemno}\vspace{2pt}\\}

%% some counters  
\newcommand{\assignment}{\arabic{assignmentno}}

%% solution  
\newcommand{\solution}{\pagebreak[2]{\bf Solution:}\\}

%% Hyperref Setup
\hypersetup{pdftitle={Homework \assignment},
  pdfsubject={\coursename},
  pdfauthor={},
  pdfcreator={},
  pdfkeywords={Computability and Complexity},
  %  pdfpagemode={FullScreen},
  %colorlinks=true,
  %bookmarks=true,
  %hyperindex=true,
  bookmarksopen=false,
  bookmarksnumbered=true,
  breaklinks=true,
  %urlcolor=darkblue
  urlbordercolor={0 0 0.7}
}

\begin{document}
\coursename \hfill Course: \courseno\\
Jacobs University Bremen \hfill \mytoday\\
Dragi Kamov and Dushan Terzikj\hfill
\vspace*{0.3cm}\\
\begin{center}
{\Large \sheettitle{} \assignment\\}
\end{center}

\exercise{}{0}
\solution
I will be using the axioms and inference rules from section 7.1 on page 48 in the lecture notes.\\ \\
\textbf{call-by-value:}
\begin{itemize}
    \item
    \begin{align*}
        K(KI(KI))y&\rightarrow KIy\text{ (A3)}\\
        KIy&\rightarrow I\text{ (A3)}\\
        K(KI(KI))y&\rightarrow I\text{ (I3)}\\
    \end{align*}
    \item
    \begin{align*}
        K(KI(KIy))y&\rightarrow K(KII)y\text{ (A3)}\\
        K(KII)y&\rightarrow KIy\text{ (A3)}\\
        KIy&\rightarrow I\text{ (A3)}\\
        K(KI(KIy))y&\rightarrow I\text{ (A3)}\\
    \end{align*}
    \item
    \begin{align*}
        SK(KK(KIy))y&\rightarrow SK(KKI)y\text{ (A3)}\\
        SK(KKI)y&\rightarrow SKKy\text{ (A3)}\\
        SKKy&\rightarrow SK\text{ (A3)}\\
    \end{align*}
    This one is a dead end.\\
\end{itemize}
\textbf{call-by-name}:
\begin{itemize}
    \item
    \begin{align*}
        K(KI(KI))y&\rightarrow KI(KI)\text{ (A3)}\\
        KI(KI)&\rightarrow I\text{ (A3)}\\
        K(KI(KI))y&\rightarrow I\text{ (I3)}\\
    \end{align*}
    \item
    \begin{align*}
        K(KI(KIy))y&\rightarrow KI(KIy)\text{ (A3)}\\
        KI(KIy)&\rightarrow I\text{ (A3)}\\
        K(KI(KIy))y&\rightarrow I\text{ (I3)}\\
    \end{align*}
    \item
    \begin{align*}
        SK(KK(KIy))y&\rightarrow Ky((KK(KIy))y)\text{ (A2)}\\
        Ky((KK(KIy))y)&\rightarrow y\text{ (A3)}\\
        SK(KK(KIy))y&\rightarrow y\text{ (I3)}\\
    \end{align*}
\end{itemize}

\newpage

\exercise{}{0}
\solution
\begin{enumerate}[label=(\alph*)]
    \item $Cxy=y$, using \textbf{call-by-name} evaluation:
    \begin{align*}
        SKxy&\rightarrow Ky(xy)\text{ (A2)}\\
        Ky(xy)&\rightarrow y\text{ (A3)}\\
        \Rightarrow C&=SK\\
    \end{align*}
    \item $Cxyz=y$ using \textbf{call-by-name} evaluation:
    \begin{align*}
        KKxyz&\rightarrow Kyz\text{ (A3)}\\
        Kyz&\rightarrow y\text{ (A3)}\\
        \Rightarrow C&=KK\\
    \end{align*}
    \item $Cxyz=x$ using \textbf{call-by-name} evaluation:
    \begin{align*}
        S(KK)Kxyz&\rightarrow KKx(Kx)yz\text{ (A2)}\\
        KKx(Kx)yz&\rightarrow K(Kx)yz\text{ (A3)}\\
        K(Kx)yz&\rightarrow Kxz\text{ (A3)}\\
        Kxz&\rightarrow x\text{ (A3)}\\
        \Rightarrow C&=S(KK)K\\
    \end{align*}
\end{enumerate}

\exercise{}{0}
\solution
The following solution was done with the help of Ana Ambroladze.\\\\
$ t_2[x,y] = y $ \\
$ t_2[x,y] = t_1[x]y = y $ \\
We can see that it is not composite so: \\
$ \implies t_1[x] = I $ \\
$ t_1[x] = t_0[$  $]x $ \\
And as that is not composite: \\
$ \implies t_0[$  $] = KI $ \\\\
$ t_2[x,y] = t_1[x]y = t_0[$  $]xy = KIxy = y $ \\
$ \implies C = KI $


\end{document}