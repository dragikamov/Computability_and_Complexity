\documentclass[a4paper]{article}
\usepackage[pdftex]{hyperref}
\usepackage[latin1]{inputenc}
\usepackage[english]{babel}
\usepackage{a4wide}
\usepackage{amsmath}
\usepackage{amssymb}
\usepackage{algorithmic}
\usepackage{algorithm}
\usepackage{ifthen}
\usepackage{listings}
% move the asterisk at the right position
\lstset{basicstyle=\ttfamily,tabsize=4,literate={*}{${}^*{}$}1}
%\lstset{language=C,basicstyle=\ttfamily}
\usepackage{moreverb}
\usepackage{palatino}
\usepackage{multicol}
\usepackage{tabularx}
\usepackage{comment}
\usepackage{verbatim}
\usepackage{color}
% \usepackage[utf8]{inputenc}
\usepackage{tabulary}
\usepackage{colortbl}

% Because of an error on line 41 I added this
\usepackage{graphicx}

% Used for drawing DFAs and NFAs
\usepackage{tikz}
\usetikzlibrary{automata, positioning}

% Tab
\newcommand\tab[1][1cm]{\hspace*{#1}}

% HalfTab
\newcommand\halftab[1][0.5cm]{\hspace*{#1}}

% Checkmark
\usepackage{amssymb}
\usepackage{pifont}
\newcommand{\cmark}{\ding{51}}%
\newcommand{\xmark}{\ding{55}}%

%% pdflatex?
\newif\ifpdf
\ifx\pdfoutput\undefined
\pdffalse % we are not running PDFLaTeX
\else
\pdfoutput=1 % we are running PDFLaTeX
\pdftrue
\fi
%\ifpdf
%\usepackage[pdftex]{graphicx}
%\else
%\usepackage{graphicx}
%\fi
\ifpdf
\DeclareGraphicsExtensions{.pdf, .jpg}
\else
\DeclareGraphicsExtensions{.eps, .jpg}
\fi

\parindent=0cm
\parskip=0cm

\setlength{\columnseprule}{0.4pt}
\addtolength{\columnsep}{2pt}

\addtolength{\textheight}{5.5cm}
\addtolength{\topmargin}{-26mm}
\pagestyle{empty}

\let\oldemptyset\emptyset
\let\emptyset\varnothing

%%
%% Sheet setup
%% 
\newcommand{\coursename}{Computability and Complexity}
\newcommand{\courseno}{CO21-320211}
 
\newcommand{\sheettitle}{Homework}
\newcommand{\mytitle}{}
\newcommand{\mytoday}{February 19, 2019}

\newcommand{\qed}{\ensuremath\square}

% Current Assignment number
\newcounter{assignmentno}
\setcounter{assignmentno}{2}

% Current Problem number, should always start at 1
\newcounter{problemno}
\setcounter{problemno}{1}

%%
%% problem and bonus environment
%%
\newcounter{probcalc}
\newcommand{\exercise}[2]{
  \pagebreak[2]
  \setcounter{probcalc}{#2}
  ~\\
  {\large \textbf{Exercise \arabic{problemno}} \hspace{0.2cm}\textit{#1}} \refstepcounter{problemno}\vspace{2pt}\\}

\newcommand{\bonus}[2]{
  \pagebreak[2]
  \setcounter{probcalc}{#2}
  ~\\
  {\large \textbf{Bonus Problem \arabic{assignmentno}.\arabic{problemno}} \hspace{0.2cm}\textit{#1}} \refstepcounter{problemno}\vspace{2pt}\\}

%% some counters  
\newcommand{\assignment}{\arabic{assignmentno}}

%% solution  
\newcommand{\solution}{\pagebreak[2]{\bf Solution:}\\}

%% Hyperref Setup
\hypersetup{pdftitle={Homework \assignment},
  pdfsubject={\coursename},
  pdfauthor={},
  pdfcreator={},
  pdfkeywords={Computability and Complexity},
  %  pdfpagemode={FullScreen},
  %colorlinks=true,
  %bookmarks=true,
  %hyperindex=true,
  bookmarksopen=false,
  bookmarksnumbered=true,
  breaklinks=true,
  %urlcolor=darkblue
  urlbordercolor={0 0 0.7}
}

\begin{document}
\coursename \hfill Course: \courseno\\
Jacobs University Bremen \hfill \mytoday\\
Dragi Kamov and Dushan Terzikj\hfill
\vspace*{0.3cm}\\
\begin{center}
{\Large \sheettitle{} \assignment\\}
\end{center}

\exercise{}{0}
\solution \\ \\
The idea is to use 2-tape TM: the first tape will be read only and it will be the input tape with the word $w$ written on it, while the second tape will be a binary counter. This "binary counter" will represent a number in binary form. \\ \\
How will the increments happen? Well whenever you increment 1, first replace all leading 1s to 0s and write the first 0 you encounter to 1. For example if we have $1110100$ on the tape and we want to increment one, it turns into $0001100$. Take a look at the next page for a TM which does that (apologize in advance for the quality of the picture, please don't judge us).\\ \\

The transitions which go from $q0$ to $q1$ and $q0$ to $q3$ are transitions made when there are still 1s in the input tape, while the transition $q0$ to $q2$ is when the decision comes. All numbers which are powers of 2 have this form $0*1$. Starting from $q2$ that is being decided: if the number on the write tape is in the form 0*1, then the input word $w$ is of length power of 2.

\newpage
\exercise{}{0}
\solution
Solution coppied from Brian Sherif. \\ \\
(a) Are the functions f(n) = exp(n) and g(n) = exp(2n) polynomially related? \\
Yes, f(n) and g(n) are polynomially related. 
After reading, we can find that there exist a p such that $f(n) <= p(g(n))$ while also being $g(n) <= p(f(n))$. \\
if we consider our p(x)=$x^2$ we find the ability to make $p(f(n)) == g(n)$ and $f(n)<=p(g(n))$.
\\

Showing indeed that they are both polynomially related\\

(b)What about f(n) = exp(n) and g(n) = exp($n^2$)?
When we consider the rate of growth for both the exp(n) and exp($n^2$).  We can calculate the growth rate by considering the slope (calculating its derivative. The derivative of exp(n) is exp(n) while the exp($n^2$) will be equal to 2n(exp($n^2$)) showing a much more aggressive growth rate. \\
To prove it further we can attempt to prove that there exist there exist no $'p' f(n) <= p(g(n))$ while also being $g(n) <= p(f(n)$).\\
We can attempt this by first trying to equate g(n) and f(n).  As lim of n to infinity we can make exp($n^2$) = exp(n) when Px = $x^2$ then P(f(n)) = g(n) as n goes to infinity. But when reversed; that is not the same. \\
On the other hand; trying to equate g(n) to f(n) by making P(x)=$x^{0.25}$ will lead to an undesired equation if we apply it to f(n) hence showing that f(n) = exp(n) and g(n) = exp($n^2$) are not polynomially related.

\end{document}